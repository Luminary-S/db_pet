% Beamer Presentation
% LaTeX Template
% Version 1.0 (10/11/12)
%
% This template has been downloaded from:
% http://www.LaTeXTemplates.com
%
% License:
% CC BY-NC-SA 3.0 (http://creativecommons.org/licenses/by-nc-sa/3.0/)
%


%----------------------------------------------------------------------------------------
%   PACKAGES AND THEMES
%----------------------------------------------------------------------------------------

\documentclass[xcolor={dvipsnames},aspectratio=169]{beamer}

\mode<presentation> {

% The Beamer class comes with a number of default slide themes
% which change the colors and layouts of slides. Below this is a list
% of all the themes, uncomment each in turn to see what they look like.

%\usetheme{default}
%\usetheme{AnnArbor}
%\usetheme{Antibes}
%\usetheme{Bergen}
\usetheme{Berkeley}
%\usetheme{Berlin}
%\usetheme{Boadilla}
%\usetheme{CambridgeUS}
%\usetheme{Copenhagen}
%\usetheme{Darmstadt}
%\usetheme{Dresden}
%\usetheme{Frankfurt}
%\usetheme{Goettingen}
% \usetheme{Hannover} %待选
%\usetheme{Ilmenau}
%\usetheme{JuanLesPins}
%\usetheme{Luebeck}
%\usetheme{Madrid}
%\usetheme{Malmoe}
%\usetheme{Marburg}
%\usetheme{Montpellier}
%\usetheme{PaloAlto}
%\usetheme{Pittsburgh}
%\usetheme{Rochester}
%\usetheme{Singapore} %待选
%\usetheme{Szeged}
%\usetheme{Warsaw}

% As well as themes, the Beamer class has a number of color themes
% for any slide theme. Uncomment each of these in turn to see how it
% changes the colors of your current slide theme.

%\usecolortheme{albatross}
%\usecolortheme{beaver}
%\usecolortheme{beetle}
%\usecolortheme{crane}
%\usecolortheme{dolphin}
%\usecolortheme{dove}
%\usecolortheme{fly}
%\usecolortheme{lily}
%\usecolortheme{orchid}
%\usecolortheme{rose}
%\usecolortheme{seagull}
%\usecolortheme{seahorse}
%\usecolortheme{whale}
%\usecolortheme{wolverine}

%\setbeamertemplate{footline} % To remove the footer line in all slides uncomment this line
%\setbeamertemplate{footline}[page number] % To replace the footer line in all slides with a simple slide count uncomment this line

%\setbeamertemplate{navigation symbols}{} % To remove the navigation symbols from the bottom of all slides uncomment this line
}

\usepackage{graphicx} % Allows including images
\usepackage{booktabs} % Allows the use of \toprule, \midrule and \bottomrule in tables
% \usepackage{ctex}
%tikz
\usepackage{tikz}
\usetikzlibrary{quotes,angles}
%\usetikzlibrary{calc,through}
\usetikzlibrary{calc,intersections,through,backgrounds}
\usetikzlibrary{shapes,snakes}
\usetikzlibrary{trees,snakes}
\usepackage{amsmath, amssymb}
\usetikzlibrary{positioning,shadows,arrows}
\usetikzlibrary{decorations.pathmorphing}
\usetikzlibrary{decorations.markings}
\usetikzlibrary{mindmap,trees}

\usepackage{ifthen}

%中文字体
\usepackage[CJKchecksingle]{xeCJK}
% \setCJKmainfont[BoldFont={WenQuanYi Zen Hei}]{WenQuanYi Zen Hei}
\setCJKmainfont{ukai.ttc}
\setCJKsansfont{ukai.ttc}


%去除下方的导航栏
\setbeamertemplate{navigation symbols}{}
%去除下方的author和title信息
%\setbeamertemplate{footline} 

% 代码高亮
% \usepackage{minted}
%----------------------------------------------------------------------------------------
%   TITLE PAGE
%----------------------------------------------------------------------------------------

\title[TikZ入门]{TikZ绘图入门} % The short title appears at the bottom of every slide, the full title is only on the title page

\author{九天学者} % Your name
\institute[募格学术] % Your institution as it will appear on the bottom of every slide, may be shorthand to save space
{
%University of California \\ % Your institution for the title page
\medskip
%\textit{john@smith.com} % Your email address
}
\date{\today} % Date, can be changed to a custom date

\begin{document}

%frame title 颜色
%\setbeamercolor{frametitle}{fg=blue,bg=white}%
%将日期隐藏
\date[]{\today}

\setbeamertemplate{itemize items}[ball]

\begin{frame}
\titlepage % Print the title page as the first slide
\centering
% \input{../Code/preface.tex}
\end{frame}

%\begin{frame}
%\frametitle{本课大纲} % Table of contents slide, comment this block out to remove it
%\tableofcontents % Throughout your presentation, if you choose to use \section{} and \subsection{} commands, these will automatically be printed on this slide as an overview of your presentation
%\end{frame}

%----------------------------------------------------------------------------------------
%   PRESENTATION SLIDES
%----------------------------------------------------------------------------------------

%------------------------------------------------
\section{简介} 

\begin{frame}{PGF \& TikZ}

什么是TikZ?什么是PGF?二者什么关系?与Latex之间什么关系?
    \begin{enumerate}
        \item 命名
        
        \textbf{PGF}: \textbf{P}ortable \textbf{G}raphics \textbf{F}ormat
        
        \textbf{TikZ}: \textbf{T}ikZ \textbf{i}st \textbf{k}ein \textbf{Z}eichenprogramm (德语)
        
        \item 关系
        
        \textbf{PGF}: 内核引擎
        
        \textbf{TikZ}: 前端宏包
        
        \item 工作环境: LaTex
        
        \item 图像输出:PostScript(dvips)和PDF (pdflatex, xelatex)
        
        \item 擅长: 几何图形和示意图(各种学科的示意图、2-D和3-D几何图形等)
        \item 不擅长:三维数据可视化
    \end{enumerate}
\end{frame}

%✂️------------------------------------------------

\begin{frame}
\Huge{\centerline{The End}}
\end{frame}

%----------------------------------------------------------------------------------------

\end{document} 
